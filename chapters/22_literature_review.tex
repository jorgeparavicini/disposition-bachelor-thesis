\chapter{Preliminary Literature Review}

Using C\# source generators to generate static code for reflected code has received little attention in the existing research. While a few books are available on the implementation of C\# source generators and their integration in Roslyn, the official documentation is lacking in detail. This limited research on the topic makes it difficult to determine the potential performance gain of using C\# source generators.

However, research on the performance penalty of reflective code in general and in other programming languages is widely available. This indicates that the potential performance benefits of using C\# source generators are significant, as reflected code is known to incur a performance penalty. This lack of existing research about C\# source generators specifically validates the need for this thesis.

The aim of this thesis is to fill this gap in the literature by investigating the potential performance gains of using C\# source generators for reflected code. Through a controlled experiment, we hope to provide a deeper understanding of the trade-off between build time and runtime performance and contribute to the field of software engineering and performance optimization.