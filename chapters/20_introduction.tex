\chapter{Introduction}

The field of software development is constantly evolving, with new technologies and tools being introduced to make the development process more efficient and streamlined. One such innovation is called C\# source generators. The C\# source generators are part of the .NET Compiler Platform and allow developers to generate source code at compile time dynamically. It works by allowing developers to write code that inspects and manipulates the syntax tree of their code, allowing them to generate additional source code that is compiled along with the rest of their code.

The history of C\# source generators dates back to the release of C\# 8.0 in 2019, which introduced the concept of source generators. Before the introduction of C\# source generators, developers had to rely on traditional code generation techniques such as T4 templates or code-behind files, which are less flexible and harder to maintain than source generators. With the release of C\# 8.0, developers now have a new tool at their disposal that can help them to improve the quality of their code and streamline their development processes.

This thesis aims to explore the potential benefits and limitations of using C\# source generators in the development of applications. The use of source generators is a relatively new and untested concept. As such, there is a gap in the existing literature regarding their efficacy and best practices for using them effectively. This thesis aims to fill that gap by empirically evaluating using C\# source generators to improve runtime performance.

The motivation behind this thesis is twofold. Firstly, there is a strong desire to improve the runtime performance of enterprise applications, as speed is a crucial factor in determining their success. This is because enterprise applications are used to manage complex and critical business processes and need to be as fast and efficient as possible to meet the demands of users. Secondly, there is a fascination with new technologies and a drive to explore and understand their potential, especially in the .Net environment. With the recent introduction of C\# source generators, there is an opportunity to evaluate their impact and determine if they have the potential to be used to generate static code for reflected code, resulting in a significant improvement in runtime performance.
