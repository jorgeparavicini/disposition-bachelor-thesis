\chapter{Introduction}

The field of software development is constantly evolving, with new technologies and tools being introduced to make the development process more efficient and streamlined. One such innovation is called C\# source generators \cite{Microsoft2022SourceGenerators}. C\# source generators are part of the .NET compiler platform \cite{Vermeir2022.NETPlatform} and allow developers to generate source code at compile time dynamically. It also allows developers to write code that inspects and manipulates the syntax tree of their code, allowing them to generate additional source code that is compiled along with the rest of their code \cite{Microsoft2022SourceGenerators, Slimak2022SourceSLIMAK, Franz2022TrendsCompilerbau}.

The history of C\# source generators dates back to the release of C\# 8.0 in 2019 \cite{Slimak2022SourceSLIMAK}, which introduced the concept of source generators. Prior to the introduction of C\# source generators, developers had to rely on traditional code generation techniques such as T4 templates \cite{Syriani2018SystematicGeneration} or code-behind files \cite{Tran2010UsingGeneration}, which are less flexible and harder to maintain than source generators. With the release of C\# 8.0, developers now have a new tool that can help them improve the quality of their code and streamline their development processes.

This thesis explores the potential benefits and limitations of using C\# source generators in application development. The use of source generators is a relatively new and untested concept. As such, there is a gap in the existing literature regarding their effectiveness and best practices for using them effectively. This thesis aims to fill this gap by empirically evaluating using C\# source generators to improve runtime performance.

The motivation for this work is twofold. First, the need for enterprise applications to be as fast and efficient as possible is a driving force behind this research. With the increasing demand for applications that can handle complex data processing, runtime performance has become an essential factor in ensuring the success of these applications. Second, I am motivated by a desire to explore and understand new technologies in the .Net environment, particularly in the area of source generators, and how they can be leveraged to improve application performance.
