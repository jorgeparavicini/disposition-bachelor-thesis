\chapter{Objective}

This thesis investigates the potential benefits of using C\# source generators to improve runtime performance by generating static code for reflected code at compile time. The research question guiding this investigation is: Can using C\# source generators to generate static code for reflected code significantly improve runtime performance? What are this approach's limitations and best practices?

In order to answer this research question, the thesis will conduct a series of experiments to measure the runtime performance of applications that use reflected code and compare the results to applications that use generated static code. This will provide an empirical evaluation of the potential benefits of using C\# source generators to improve runtime performance. In addition, the thesis will also explore the limitations and drawbacks of using this approach and identify any best practices for using C\# source generators effectively.

The aim of this research is to provide a comprehensive understanding of the potential of C\# source generators as a tool for improving application performance. By exploring the trade-offs between using reflection at runtime and generating static code at compile time, this research aims to provide insights into the future of software development and how developers can take advantage of new features in programming languages to build better, faster applications.

This research's scope is limited to using C\# source generators to generate static code for mirrored code. The research will focus on the runtime performance of applications but will not consider other factors, such as memory usage or code complexity. The research will also not cover other approaches to improving runtime performance, such as optimization techniques or alternative programming languages. Moreover, the scope will be limited to exploring the potential of C\# source generators to improve runtime performance and will not explore other potential uses for this feature. Finally, the study will be limited to applications written in C\# and will not consider applications written in other programming languages.